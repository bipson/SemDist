% This is LLNCS.DOC the documentation file of
% the LaTeX2e class from Springer-Verlag
% for Lecture Notes in Computer Science, version 2.4
\documentclass{llncs}
\usepackage{makeidx}  % allows for indexgeneration
%
\begin{document}
%
%\frontmatter          % for the preliminaries
%
\pagestyle{headings}  % switches on printing of running heads
\addtocmark{Cloud Computing} % additional mark in the TOC
%
\chapter*{Preface}
%
TODO: ADD OWN PREFACE

This textbook is intended for use by students of physics, physical
chemistry, and theoretical chemistry. The reader is presumed to have a
basic knowledge of atomic and quantum physics at the level provided, for
example, by the first few chapters in our book {\it The Physics of Atoms
and Quanta}. The student of physics will find here material which should
be included in the basic education of every physicist. This book should
furthermore allow students to acquire an appreciation of the breadth and
variety within the field of molecular physics and its future as a
fascinating area of research.

For the student of chemistry, the concepts introduced in this book will
provide a theoretical framework for that entire field of study. With the
help of these concepts, it is at least in principle possible to reduce
the enormous body of empirical chemical knowledge to a few basic
principles: those of quantum mechanics. In addition, modern physical
methods whose fundamentals are introduced here are becoming increasingly
important in chemistry and now represent indispensable tools for the
chemist. As examples, we might mention the structural analysis of
complex organic compounds, spectroscopic investigation of very rapid
reaction processes or, as a practical application, the remote detection
of pollutants in the air.

%
\tableofcontents
%
\mainmatter              % start of the contributions
%
\title{From low-level Metrics to high level Service Level Agreements}
%
\titlerunning{Cloud Computing}  % abbreviated title (for running head)
%                                     also used for the TOC unless
%                                     \toctitle is used
%
\author{Hubert~Hirsch \and Frieder~Ulm \and
Philipp~Raich}
%
\authorrunning{Hubert Hirsch et al.} % abbreviated author list (for running head)
%
%%%% list of authors for the TOC (use if author list has to be modified)
\tocauthor{Hubert Hirsch, Frieder Ulm, and Philipp Raich}
%
\institute{TU Wien, 1040 Wien, Austria,\\
\email{I.Ekeland@princeton.edu},\\ WWW home page:
\texttt{http://users/\homedir iekeland/web/welcome.html}}

\maketitle              % typeset the title of the contribution

\section{Introduction}

\subsection{What is Cloud Computing}

\subsection{SLA Overview}

\section{SLA Monitoring Approaches}

\subsection{Divide and Conquer with MAPE}

\subsection{Monitor}

\subsection{Analzye}

\subsection{Plane}

\subsection{Execute}

\subsubsection{Non-MAPE appraoches?}

\section{Conclusion}

%
% ---- Bibliography ----
%
\begin{thebibliography}{}
%
\bibitem[1980]{2clar:eke}
Clarke, F., Ekeland, I.:
Nonlinear oscillations and
boundary-value problems for Hamiltonian systems.
Arch. Rat. Mech. Anal. 78, 315--333 (1982)

\bibitem[1981]{2clar:eke:2}
Clarke, F., Ekeland, I.:
Solutions p\'{e}riodiques, du
p\'{e}riode donn\'{e}e, des \'{e}quations hamiltoniennes.
Note CRAS Paris 287, 1013--1015 (1978)

\bibitem[1982]{2mich:tar}
Michalek, R., Tarantello, G.:
Subharmonic solutions with prescribed minimal
period for nonautonomous Hamiltonian systems.
J. Diff. Eq. 72, 28--55 (1988)

\bibitem[1983]{2tar}
Tarantello, G.:
Subharmonic solutions for Hamiltonian
systems via a $\bbbz_{p}$ pseudoindex theory.
Annali di Matematica Pura (to appear)

\bibitem[1985]{2rab}
Rabinowitz, P.:
On subharmonic solutions of a Hamiltonian system.
Comm. Pure Appl. Math. 33, 609--633 (1980)

\end{thebibliography}
%\clearpage
%\addtocmark[2]{Author Index} % additional numbered TOC entry
%\renewcommand{\indexname}{Author Index}
%\printindex
%\clearpage
%\addtocmark[2]{Subject Index} % additional numbered TOC entry
%\markboth{Subject Index}{Subject Index}
%\renewcommand{\indexname}{Subject Index}
%\input{subjidx.ind}
\end{document}
